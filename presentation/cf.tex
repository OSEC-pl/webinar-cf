%
% Copyright (c) 2017 Radoslaw Kujawa.
% All rights reserved.
%
% Redistribution and use in source and binary forms, with or without
% modification, are permitted provided that the following conditions
% are met:
%
% 1. Redistributions of source code must retain the above copyright
%    notice, this list of conditions and the following disclaimer.
% 2. Redistributions in binary form must reproduce the above copyright
%    notice, this list of conditions and the following disclaimer in the
%    documentation and/or other materials provided with the distribution.
%
% THIS SOFTWARE IS PROVIDED BY RADOSLAW KUJAWA (THE AUTHOR) AND CONTRIBUTORS
% ``AS IS'' AND ANY EXPRESS OR IMPLIED WARRANTIES, INCLUDING, BUT NOT LIMITED
% TO, THE IMPLIED WARRANTIES OF MERCHANTABILITY AND FITNESS FOR A PARTICULAR
% PURPOSE ARE DISCLAIMED.  IN NO EVENT SHALL THE AUTHOR OR CONTRIBUTORS
% BE LIABLE FOR ANY DIRECT, INDIRECT, INCIDENTAL, SPECIAL, EXEMPLARY, OR
% CONSEQUENTIAL DAMAGES (INCLUDING, BUT NOT LIMITED TO, PROCUREMENT OF
% SUBSTITUTE GOODS OR SERVICES; LOSS OF USE, DATA, OR PROFITS; OR BUSINESS
% INTERRUPTION) HOWEVER CAUSED AND ON ANY THEORY OF LIABILITY, WHETHER IN
% CONTRACT, STRICT LIABILITY, OR TORT (INCLUDING NEGLIGENCE OR OTHERWISE)
% ARISING IN ANY WAY OUT OF THE USE OF THIS SOFTWARE, EVEN IF ADVISED OF THE
% POSSIBILITY OF SUCH DAMAGE.
%
% 
\documentclass[dvipsnames,table]{beamer}
\usepackage{polski}

\usetheme{Rochester}
\usecolortheme{orchid}

\usepackage{listings}
\usepackage{ucs}
\usepackage[utf8x]{inputenc}
\usepackage{wasysym}
\usepackage[normalem]{ulem}
\usepackage{amsmath}
\usepackage{hyperref}
\usepackage{tikzsymbols}

\setbeamertemplate{navigation symbols}{}
\setbeamertemplate{caption}[numbered]
\setbeamerfont{caption}{size=\scriptsize}
\setbeamercolor{framenote}{bg=OSEC-red!25}
\setbeamercolor{rednote}{bg=Red!25}
\setbeamercolor{palette primary}{use=structure,fg=white,bg=OSEC-red}
\setbeamercolor{palette secondary}{use=structure,fg=white,bg=OSEC-red2}

\setbeamertemplate{itemize item}{\scriptsize\raise1pt\hbox{\donotcoloroutermaths$\blacktriangleright$}}
\setbeamertemplate{itemize subitem}{\tiny\raise1pt\hbox{\donotcoloroutermaths$\bullet$}}
\setbeamertemplate{itemize subsubitem}{\tiny\raise1pt\hbox{\donotcoloroutermaths{--}}}

\setbeamertemplate{enumerate item}{\insertenumlabel.}
\setbeamertemplate{enumerate subitem}{\insertenumlabel.\insertsubenumlabel}
\setbeamertemplate{enumerate subsubitem}{\insertenumlabel.\insertsubenumlabel.\insertsubsubenumlabel}
\setbeamertemplate{enumerate mini template}{\insertenumlabel}

\setbeamercolor{itemize item}{fg=OSEC-red, bg=OSEC-red}
\setbeamercolor{itemize subitem}{fg=OSEC-red, bg=OSEC-red}
\setbeamercolor{itemize subsubitem}{fg=OSEC-red, bg=OSEC-red}

\setbeamercolor{section number projected}{fg=white,bg=OSEC-red}
\setbeamercolor{subsection number projected}{fg=white,bg=OSEC-red}
\setbeamercolor{button}{bg=OSEC-red,fg=white}

\setbeamertemplate{section in toc}[circle]
\setbeamertemplate{subsection in toc}[square]

\definecolor{OSEC-red}{RGB}{160,29,44}
\definecolor{OSEC-red2}{RGB}{177,76,12}
\hypersetup{colorlinks=true,linkcolor=white,urlcolor=OSEC-red}

\setlength{\tabcolsep}{8pt}
\renewcommand{\arraystretch}{1.2}

\newcommand{\tri}{$\triangleright$ }

\title{Zarządzanie Red Hat Virtualization \\ z użyciem CloudForms}
\author{Radosław Kujawa -- radoslaw.kujawa@osec.pl}
\institute{OSEC}

\begin{document}

\begin{frame}
	\titlepage
\end{frame}

\begin{frame}
\frametitle{Bardzo krótkie wprowadzenie do CloudForms}
\begin{itemize}
	\item Narzędzie do zarządzania środowiskiem IT.
	\item Spójny frontend dla całej infrastruktury wirtualizacji oraz chmury. 
	\item Self service dla użytkowników.
	\item Projekt upstream: \url{http://ManageIQ.org}.
	\item Siła CloudForms leży w jego rozszerzalności!
	\begin{itemize}
		\item Zarówno na poziomie danych jakie może prezentować\ldots
		\item Zmian w infrastrukturze jakie może wprowadzać\ldots
		\item Jak i w możliwości integracji z systemami backendowymi\ldots
	\end{itemize}	
	\item Od czegoś trzeba zacząć: jak wdrożyć CloudForms na RHV w godzinę lub mniej. \Smiley 
\end{itemize}
\end{frame}

\begin{frame}
\frametitle{Instalacja CloudForms}
\begin{itemize}
	\item Obraz CloudForms dostępny jest jako appliance.
	\begin{itemize}
		\item {\tt ovirt-image-uploader -N cfme -e export -m upload cfme-rhevm-5.7.1.3-1.x86\_64.rhevm.ova}
	\end{itemize}
	\item Import szablonu.
	\item Nowa maszyna wirtualna z zaimportowanego szablonu.
	\begin{itemize}
		\item Interfejs sieciowy.
		\item Dodatkowy dysk na bazę danych.
	\end{itemize}
	\item Start maszyny wirtualnej.
	\item Logowanie do appliance: {\tt root} / {\tt smartvm}.
\end{itemize}
\end{frame}

\begin{frame}
\frametitle{Konfiguracaja CloudForms}
\begin{itemize}
	\item Konsola zarządzania: {\tt appliance\_console}.
	\item Rekonfiguracja sieci na adresację statyczną (na screenie lub przez konsolę maszyny wirtualnej).
	\item Zmiana strefy czasowej.
	\item Synchronizacja czasu z serwerami NTP.
	\item Ustawienie nazwy hosta.
	\item Konfiguracja bazy danych.
	\begin{itemize}
		\item Dodatkowy dysk dodany do maszyny wirtualnej dla bazy danych.
	\end{itemize}
	\item Logowanie do interfejsu webowego -- bezpośrednio pod adresem appliance'u\ldots \url{http://cf.demo.osec.pl/}
	\item Rejestracja i subskrypcja w Red Hat.
\end{itemize}
\end{frame}

\begin{frame}
\frametitle{Dodawanie infrastruktury do CloudForms}
\begin{itemize}
	\item Dodajmy naszą instalację RHV do CloudForms!
	\begin{itemize}
		\item Compute \tri Infrastructure \tri Providers.
		\item Configuration \tri Add a new infrastructure provider.
		\item Type \tri Red Hat Virtualization Manager.
		\item Hostname, username, password\ldots
	\end{itemize}
\end{itemize}
\end{frame}

\begin{frame}
\frametitle{Provisioning maszyn wirtualnych}
\begin{itemize}
	\item Compute \tri Infrastructure \tri Virtual Machines.
	\item Lifecycle \tri Provision VMs.
	\item Wybieramy template\ldots
	\item Request \tri E-mail, First Name, Last Name.
	\item Catalog \tri VM Name.
\end{itemize}
\end{frame}

\begin{frame}
\frametitle{Raportowanie}
\begin{itemize}
	\item Cloud Intel \tri Reports.
	\item Raporty w oparciu o dane z providerów którymi CloudForms zarządza. 
\end{itemize}
\end{frame}

\begin{frame}
\frametitle{Przyjdź na barcamp OSEC}
\begin{itemize}
	\item 4 Kwietnia 2017, Warszawa, Restauracja Kameralna.
	\item Gość specjalny: Carol Chen -- Community Development Manager ManageIQ/CloudForms at Red Hat.
	\item Prezentacja i demo na żywo ManageIQ -- dowiedz się więcej o jego możliwościach.
	\item Wstęp bezpłatny.
	\item \url{https://osec.pl/barcamp/barcamp-kwiecien-2017}
	\item Zapisy: \href{mailto:barcamp@osec.pl}{barcamp@osec.pl} .
\end{itemize}
\end{frame}

\begin{frame}
\frametitle{Dziękuję!}
\begin{itemize}
	\item Jeśli macie dodatkowe pytania...
	\item radoslaw.kujawa@osec.pl
\end{itemize}
\end{frame}

\end{document}

